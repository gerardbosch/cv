%% Copyright 2006-1008 Xavier Danaux (xdanaux@gmail.com).
%
% This work may be distributed and/or modified under the
% conditions of the LaTeX Project Public License version 1.3c,
% available at http://www.latex-project.org/lppl/.

\documentclass[10pt,a4paper,colorlinks=true]{moderncv} % includes Font Awesome
% optional argument are 'blue' (default), 'orange', 'red', 'green', 'grey' and 'roman' (for roman fonts, instead of sans serif fonts)
% \moderncvtheme[blue, roman]{casual}
\moderncvtheme[blue]{classic}
\renewcommand*{\labelitemi} {\strut\textcolor{color1}{\tiny\faCircle}} % default: \faCircle[regular] (regular style)
%\faStyle{regular} % {regular, solid} default: regular. Font Awesome package included by moderncv

\usepackage[english]{babel}

\usepackage{lmodern} % Latin Modern fonts, vectorized improvement of Computer Modern
\usepackage{emoji}
\usepackage{changepage} % for the adjustwidth environment

% Add a timeline on the left
\usepackage[firstyear=2010, lastyear=2022]{moderntimeline}
\setlength{\hintscolumnwidth}{4.1cm} % width of the timeline on your left

% Globally modify itemsep on itemize environment
\usepackage{enumitem}
\setitemize{itemsep=4pt}

% Adjust the page margins: Easy layout setup
%\addtolength{\voffset}{1cm}
\usepackage[scale=0.8]{geometry}
\addtolength{\textheight}{1cm}
%\setlength{\hintscolumnwidth}{3cm}						% if you want to change the width of the column with the dates
%\AtBeginDocument{\setlength{\maketitlenamewidth}{6cm}} % only for the classic theme, if you want to change the width of your name placeholder (to leave more space for your address details
\AtBeginDocument{\recomputelengths}                     % required when changes are made to page layout lengths

\usepackage{hanging} % hanging paragraph

\nopagenumbers{} % uncomment to suppress automatic page numbering for CVs longer than one page
% Page numbering provided by moderncv does no longer work atm, so manually add it
\usepackage{lastpage}
\rfoot{\addressfont\itshape\textcolor{gray}{Page \thepage\ / \pageref*{LastPage}}}
\lfoot{\href{https://github.com/gerardbosch/cv}{\textcolor{gray}{This CV is made with \LaTeX~/ built by \faGithub*~GitHub Actions / delivered via \faCloudflare~Cloudflare}}}

%----------------------------------------------------------------------------------
%  Fork me on GitHub
%----------------------------------------------------------------------------------

\usepackage{tikz}
\usepackage{ifthen}
\usepackage{xcolor}
\usetikzlibrary{shadows.blur}

\definecolor{forkmebg}{HTML}{3873b3}
\definecolor{forkmefg}{HTML}{eeeeee}

% rotate and shift are hardcoded here because of the href workaround, see also https://gist.github.com/dokenzy/fe8967c8a5921191a261
\newcommand{\forkme}{
  \begin{tikzpicture}[remember picture, overlay]
    \node[rotate=45, shift={(0, -3.2cm)}] at (current page.north west) {
      \begin{tikzpicture}[remember picture, overlay]
        \node[fill=forkmebg, text centered, minimum width=50em, minimum height=3.0em, blur shadow, shadow yshift=0pt, shadow xshift=0pt, shadow blur radius=.4em, shadow opacity=50](fmogh) at (0pt, 0pt) {}; % workaround: text/link see below
        \draw[forkmefg!60, dashed, line width=.08em, dash pattern=on .5em off 1.5\pgflinewidth] (-25em,1.2em) rectangle (25em,-1.2em);
      \end{tikzpicture}
    };
    % workaround: the link needs to be defined using a \rotatebox{deg} because \node[rotate=deg]{\href} isn't rendered correctly
    \node[shift={(0.25cm, 3.3cm)}, text centered, minimum width=50em, minimum height=3.0em, text=forkmefg](fmogh) at (0pt, 0pt) {
      \href{https://github.com/gerardbosch/cv}{\rotatebox{45}{\setmainfont{Inter}\bfseries\color{white}{Fork me on GitHub}}}
    };
  \end{tikzpicture}
}

%----------------------------------------------------------------------------------
% Personal data
%----------------------------------------------------------------------------------

\firstname{Gerard}
\familyname{Bosch}
\title{Curriculum Vitae}               % optional, remove the line if not wanted
\address{\faMapMarker~ Barcelona, Spain}    % optional, remove the line if not wanted
%\mobile{}                                 % optional, remove the line if not wanted
%\phone{}                                   % optional, remove the line if not wanted
%\fax{fax (optional)}                       % optional, remove the line if not wanted
\email{gerard.bosch@gmail.com}              % optional, remove the line if not wanted

\extrainfo{
  \faLinkedin~ \href{https://www.linkedin.com/in/gerard-bosch}{linkedin.com/in/gerard-bosch}\\
  \faGithub~ \href{https://github.com/gerardbosch}{github.com/gerardbosch}
}

\photo[64pt]{pic2.jpg}                         % '64pt' is the height the picture must be resized to and 'picture' is the name of the picture file; optional, remove the line if not wanted

% Optional, remove the quote if not wanted
% Don't know why, but using T1 encoding for quote improves font quality on some letters (lualatex uses TU encoding)
\quote{\fontencoding{T1}\selectfont
  ``The purpose of abstraction is not to be vague, but to create a new semantic level in which one can be absolutely precise.''\\[5pt]
  \parbox{\textwidth}{\hfill---Edsger W. Dijkstra}
}
% \quote{\fontencoding{T1}\selectfont
%  ``All problems in computer science can be solved\\
%   by another level of indirection''
%   \parbox{\textwidth}{\hfill---Butler Lampson}
% }


%----------------------------------------------------------------------------------
% Content
%----------------------------------------------------------------------------------
\begin{document}

\forkme% % keep this line together with \maketitle (with no line breaks) to avoid the ribbon text to be misplaced :-/
\maketitle

% Increase cventry vertical space
\setlength{\parskip}{4pt}

% Live CV note
\vspace{-2ex}\centerline{
  \href{https://cv.gerardbosch.xyz}{\emoji{person-gesturing-ok} \textcolor{black}{This CV is \textcolor{magenta}{live!} Click here to ensure you access the most \textbf{up-to-date} version \emoji{person-gesturing-ok}}}
}\vspace{4ex}

[ \emph{Read the one-page résumé version at} \href{https://resume.gerardbosch.xyz}{https://resume.gerardbosch.xyz} ]

% \vspace{-0.7cm}
% \section{\textsc{Personal Information}}
% \cvline{Name}{Gerard Bosch}
% \cvline{Nationality}{}
% \cvline{Birth date}{}
% \cvline{Address}{}
% \cvline{}{}
% \cvline{Phone}{}
% \cvline{e-mail}{}

\section{\textsc{Summary}}
\vspace{1em}
\begin{adjustwidth}{1cm}{1.9cm}
Enthusiast about Functional Programming, software engineering/craftsmanship, clean code, architectural/design patterns and programming languages theory.

My background is mostly about designing/implementing backend systems and APIs. I have good analysis skills and can break down requirements well and transform that into code.

Really concerned about the code itself, its readability, expressiveness, precision and conciseness. Continuous improvement advocate \& continuous refactor and code reuse as a mantra. Accidental-complexity fighter. Curious by nature, proactive and self-taught. Like learning~\emoji{smiley}, love crafting \emoji{heart}.
\end{adjustwidth}
\vspace{1em}

\section{\textsc{Presentations}}

%\cventry{year--year}{Job title}{Employer}{City}{}{Description}                % arguments 3 to 6 are optional
%\cventry{year--year}{Job title}{Employer}{City}{}{Description line 1\newline{}Description line 2}% arguments 3 to 6 are optional

\cventry{2020}{\href{https://bit.ly/fp-short-intro}{\textnormal{Functional Programming super-short intro}}}{}{}{}{}
\cventry{2018}{\href{https://blockchain-presentation.gerardbosch.xyz}{\textnormal{A technical introduction to Blockchain technology}}}{}{}{}{}


\vspace{.3cm}
\section{\textsc{Professional Experience}}

% \renewcommand{\labelitemi}{$-$}

\tlcventry{2021}{0}{Software Engineer}{N26 Bank}{Barcelona}{}{
  Joined the fees team, which is in charge of designing and operating a new fee platform intended to be the new-standardized way of charging company-wide banking fees.
  \begin{itemize}\setlength\itemsep{4pt}
    \item That implied not only the creation of a dedicated microservice, also re-purposing and refactoring of existent ones along with APIs, as well as DB migrations.
    \item Implement upcoming business use cases as per new fee creation or regulatory compliance.
    \item Being a bank, testing is one of the most important and cared matters (pyramid testing, contract testing, regression, TDD,\ldots).
  \end{itemize}
  \hangpara{1.5em}{1}\textbf{Keywords}: \textit{Kotlin, Spring Boot, Microservices, EDA, Kafka, REST, Postgres, Datadog, ELK, Kibana, AWS, Hexagonal, Clean Architecture, Domain Driven Design.}
}

\tlcventry{2016}{2021}{Senior Software Engineer}{GFT IT Consulting}{Lleida/Barcelona}{}{
  \vspace{-5pt}
  PROJECT HIGHLIGHTS (\textit{most relevant projects only}):
  \begin{itemize}\setlength\itemsep{4pt}
     \item \textit{Banc Sabadell}: Middleware Architecture (2020) \\[2pt]
           \textbf{Role}: Software Engineer \\[2pt]
           Start redefining from scratch a whole new framework and development environment based on MsA, Spring Boot, API First, Event Driven, DDD, Hexagonal and so forth.
     \item \textit{Banc Sabadell}: PSD2 \& APIfication (2018--2020) \\[2pt]
           \textbf{Role}: Tech Lead (development team) \\[2pt]
           • Design and implement the APIs for the European regulatory PSD2 project, leading the first approach to API exposition for the Banc Sabadell. I worked both on the API definition side and the Architecture side, defining and implementing architectural components for the Microservices Architecture where I was established as the technical lead for the development team until post-production. \\[2pt]
           • Defined and set up an \emph{API-First} approach with OpenAPI and API artifact (\textsc{dto}) generation with
             CI integration, making API definition agnostic and decoupled from its implementation.
     \item \textit{Bankinter}: Microservice architecture (2017) \\[2pt]
           \textbf{Role}: Developer \\[2pt]
           Contribute in the development of an architecture for Bankinter based on MsA (Micro Services Architecture) with Java 8 and Spring Boot/Cloud. The built Architecture provides the pre-configured components: from security, logging, audit,\ldots~to cryptography; to allow developers to easily get started implementing microservices with not much Spring background required.
   \end{itemize}
   \hangpara{1.5em}{1}\textbf{Keywords}: \textit{Spring, Java, Vavr, FP, OpenShift, ELK, Kibana, OpenAPI, API-First, Blockchain.}
}

\tlcventry{2014}{2016}{Software Engineer}{ICG Software}{Torrefarrera, Lleida}{}{
  Senior developer in charge of mobile applications for retail.
\begin{itemize}
  \item Port a whole Android retail selling application to Windows Phone platform due to customer demands. In the way, learn both Android and WP frameworks.
  \item Build from scratch a new application for warehouse and inventory management; featuring integrated in-app barcode scanner through phone's camera, Android object database, and dual online/offline working mode.
  \item Research and use of cutting-edge technologies and libraries (Android data binding, Realm databases, Gson parser, Retrofit\ldots).
  \item Architectural and design patterns implementation.
\end{itemize}
\hangpara{1.5em}{1}\textbf{Keywords}: Java, C\#/.NET, Mobile apps, Android, Windows Phone, Realm, REST.
}


\tlcventry{2011}{2013}{Researcher/developer}{Artificial Intelligence Research Institute (IIIA)-Spanish National Research Council (CSIC)}{Bellaterra}{}{Hired for the research project Innpacto-2011 (Spanish Science and Innovation Ministry), \emph{NEWMATICA: Intelligent and Energy Efficient Advanced System for Vacuum Waste Collection} (IPT-2011-1496-310000). In collaboration with University of Lleida.
%\renewcommand{\labelitemi}{$-$}
\begin{itemize}
 \item Research on artificial intelligence algorithms and machine learning.
 \item Design and implementation of a simulator, modelling of real plants topology and operation. Experimentation running bulk simulations in a cluster and automatic processing and representation of results.
 \item Implementation of algorithms based on \emph{Approximate Dynamic Programming} (Reinforcement Learning) in order to optimize operational plans of waste collecting plants.
\end{itemize}
\hangpara{1.5em}{1}\textbf{Keywords}: Approximate Dynamic Programming, ADP, Discrete Event Simulation, DES, Integer Linear Programming, ILP, modelling, data collection, AI, research, planning, optimization, Python.
}

\tlcventry{2010}{2011}{Systems/comms operations}{Information Systems and Communications Area, University of Lleida}{Lleida}{}{
%\renewcommand{\labelitemi}{$-$}
\begin{itemize}
\item Setting up and deployment of Open-Exchange system (\emph{Open Source}) covering the need of a corporate \emph{Groupware} for the university staff. Set up, deployment, integration and maintenance of the service. Integration of the system with the corporate mobile phones.
\item Setting up of an open source system for the management of name servers (DNS) in the university.
\item Installation and management of GNU/Linux servers and virtual machines.
\end{itemize}
\hangpara{1.5em}{1}\textbf{Keywords:} Linux, virtualization, servers, DNS, networking, groupware
}


\section{\textsc{Technical Skills}}

%\cvcomputer{category 1}{XXX, YYY, ZZZ}{category 4}{XXX, YYY, ZZZ}
%\cvcomputer{Operating Systems}{MS Windows, GNU/Linux (installation, administration, configuration and server administration.)}
%{Networks}{Architectures and net protocols: TCP, UDP, HTTP, DNS, WLAN; routing, firewalling\ldots\\
%Equipment and services configuration. \\
%Distributed environments programming, RMI, Web Services.
%}

\cvcomputer{Programming languages}{Java, Scala, Kotlin, Bash, Haskell, Python, Perl, SQL.}
           {Others}{Git, Maven/Gradle/sbt, JGitVer, Docker, OpenShift, Cloudflare, \LaTeX, Linux, Nix, GNU utils, Cloud Computing.}

\cvcomputer{Development methodologies}{FP patterns, GoF design patterns, SOLID, Clean Code, Hexagonal/Clean Architecture, Agile development, SCRUM, Domain Driven Design, TDD.}
           {Interest Areas}{Functional Programming, Big/Fast Data \& Streaming, DLTs \& Blockchain, Artificial Intelligence.}


\section{\textsc{Courses and certifications}}
\cventry{2020}{Big Data Foundations}{IBM Cognitive Class}{\href{https://www.youracclaim.com/badges/70dcdd97-4307-460a-8228-9a39a99b05de}{Course badge}}{}{}
\cventry{2020}{Hadoop Foundations}{IBM Cognitive Class}{\href{https://www.youracclaim.com/badges/0bf1199a-0e3e-40a1-8bb8-bb1e765eaf64}{Course badge}}{}{}
\cventry{2020}{Spark}{IBM Cognitive Class}{\href{https://www.youracclaim.com/badges/650e30f0-5b7b-49f4-ba74-4a957577b8a7}{Course badge}}{}{}
\cventry{2020}{Big Data Foundations (level 2)}{IBM Cognitive Class}{\href{https://www.youracclaim.com/badges/4a378986-4d19-41a8-8113-0099b923ef65}{Course badge}}{}{}
\cventry{2018}{Serverless Data Analysis with Google BigQuery and Cloud Dataflow}{Coursera}{\href{https://www.coursera.org/account/accomplishments/verify/7Y7BA7B6E6PE}{Course accomplishment}}{}{}
\cventry{2018}{Leveraging Unstructured Data with Cloud Dataproc on Google Cloud Platform}{Coursera}{\href{https://www.coursera.org/account/accomplishments/verify/9KHX6ZSTUDHY}{Course accomplishment}}{}{}
\cventry{2018}{Google Cloud Platform Big Data and Machine Learning Fundamentals}{Coursera}{\href{https://www.coursera.org/account/accomplishments/verify/BCSRMUPXQZB7}{Course accomplishment}}{}{}

\section{\textsc{Conference Attendance}}
\cventry{2020--2021}{\textnormal{Several online conferences, such as} JLove, Scala Love in the City, Haskell Love,\ldots}{online}{}{}{}
\cventry{2019}{API Days}{Paris}{}{}{}
\cventry{2018}{API Days}{Paris}{}{}{}
\cventry{2018}{Ethereum Community Conference}{Paris}{}{}{}
\cventry{2017}{JBcnConf}{Java \& JVM Conference}{Barcelona}{}{}

\newpage
\section{\textsc{Education}}
%\cventry{year--year}{Degree}{Institution}{City}{\textit{Grade}}{Description}  % arguments 3 to 6 are optional
\cventry{2011}{Master in Open Source Software Engineering}{University of Lleida}{Lleida / \textit{VIA University College} (Denmark)}{}{}
\cventry{2011}{IT Engineering Higher Degree}{University of Lleida}{Lleida / \textit{VIA University College} (Denmark)}{}{}
\cventry{2005--2010}{IT Engineering Bachelor}{University of Lleida}{Lleida}{}{}


\section{\textsc{Final Degree Dissertations}}
\subsection{Master in Open Source Software Engineering}
%\cvline{title}{\emph{Title}}
%\cvline{supervisors}{Supervisors}
%\cvline{description}{\small Short thesis abstract}
\cvline{title}{\emph{Setting up and deployment of Sauron system for DNS system management at University of Lleida.}}
\cvline{date}{28--09--2011}
\cvline{advisor}{Dr. Carles Mateu Piñol, Department of Computer Science and Industrial Engineering, University of Lleida, Lleida.}
\cvline{description}{Setting up an open source system for the name servers (DNS) management of the university. Integration of several open source components together with an LDAP database into ISC BIND name server. Implementation, configuration, integration and deployment of that system in the university's production infrastructure.}
\cvline{}{\textsl{\textbf{Qualified} with honors.}}
\cvline{document link}{\href{https://repositori.udl.cat/bitstream/handle/10459.1/45832/Bosch.pdf}{https://repositori.udl.cat/bitstream/handle/10459.1/45832/Bosch.pdf}}
\cvline{}{\footnotesize{(Catalan + English full configuration guide)}}
\cvline{presentation (English)}{\scalebox{.7}{\url{http://www.slideshare.net/rockingerard/sauron-system-implementation-and-deployment-for-dns-management}}}


\subsection{IT Engineering Bachelor}
%\cvline{title}{\emph{Title}}
%\cvline{supervisors}{Supervisors}
%\cvline{description}{\small Short thesis abstract}

\cvline{title}{\emph{Implementation of Golay codes in SAGE.}}
\cvline{date}{15--01--2010}
\cvline{advisor}{Dr. Ramiro Moreno Chiral, Department of Mathematics, University of Lleida, Lleida.}
\cvline{description}{Implementation of a kind of channel error correcting codes (ECC) inside mathematical Open Source Software package SAGE, which is built up in Python and just previously had a generic implementation for linear codes.}
\cvline{}{\textsl{\textbf{Qualified} with honors.}}
\cvline{document link}{\href{https://repositori.udl.cat/bitstream/handle/10459.1/45751/Bosch.pdf}{https://repositori.udl.cat/bitstream/handle/10459.1/45751/Bosch.pdf}}
\cvline{presentation}{\scalebox{.85}{\url{http://www.slideshare.net/rockingerard/implementaci-dels-codis-de-golay-en-sage}}}



\section{\textsc{Languages}}
\cvlanguage{English}{\parbox[t]{\textwidth}{Professional working proficiency\\
                                            \textsl{\small{B2 CEFR (Common European Framework of Reference)}}}}{}
\cvlanguage{Catalan}{Mother tongue}{}
\cvlanguage{Spanish}{Mother tongue}{}

%\section{Interests}
%\cvline{hobby 1}{\small Description}
%\cvline{hobby 2}{\small Description}
%\cvline{hobby 3}{\small Description}

\vspace{1.5em}
\begin{flushright}
  \emph{The current copy was last updated on} \today.\\*  % new line preventing page break
  \href{https://cv.gerardbosch.xyz}{\emph{Check last version here}}
\end{flushright}



%\section{Extra 1}
%\cvlistitem{Item 1}
%\cvlistitem{Item 2}
%\cvlistitem[+]{Item 3}            % optional other symbol

%\section{Extra 2}
%\cvlistdoubleitem[\Neutral]{Item 1}{Item 4}
%\cvlistdoubleitem[\Neutral]{Item 2}{Item 5}
%\cvlistdoubleitem[\Neutral]{Item 3}{}

% Publications from a BibTeX file
%\nocite{*}
%\bibliographystyle{plain}
%\bibliography{publications}       % 'publications' is the name of a BibTeX file

\end{document}
